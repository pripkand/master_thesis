10-dimensional TypeIIA string theory in the strongly coupled regime can be identified with an 11-dimensional theory named M\footnote{M for Mother, Matrix, Magic, Membrane, Mystery etc.} Theory\cite{Mthoery}. The identification comes from the fact that, were this to be the case, the string coupling would be related to the size of the extra dimension and so in the case where the theory was compactified (weak coupling) the coupling would be finite while as the coupling increases the size of the extra dimension would as well resulting in the decompactification of the 11$^{th}$ dimension.

Importantly, the objects postulated to populate M Theory (M2 branes and M5 branes) produce most of the known objects of TypeIIA string theory such as D2-branes and, more importantly for what is to follow, D0-branes which come from the massive modes that correspond to the compactified dimension. Since the number of branes corresponds to massive Kaluza-Klein (KK) modes, accessing a limit where we have $N\rightarrow\infty$ such modes we access the uncompactification limit and thus M theory. We will now focus on how to describe N D0-branes in this limit.