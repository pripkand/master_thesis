For reviews on the Matrix model, refer to \cite{Taylor,Matrix}


We will now list some known properties of the matrix model.
\paragraph{Coordinate Interpretation of X}
Let us start by focusing on the interpretation of the $X^i$ matrices as the coordinates of the D0-branes. More precisely, if we have NxN matrices then the 9 $X^i$ matrices' N eigenvalues describe the N 9-dimensional position vectors of the N D0-Branes, i.e. the $n^{\text{th}}$ eigenvalue of $X^i$ is the $i^{\text{th}}$ component of the $n^{\text{th}}$ brane\cite{Matrix,Taylor}. 

The off diagonal terms relate the interactions between different branes which increase in importance as the branes get closer together since the open strings that end on them stretch between different branes \cite{Matrix}. The trace of each matrix corresponds to the centre of mass coordinates. If we instead worked with traceless X matrices, then the Hamiltonian would have an SU(N) symmetry instead of U(N)\cite{Matrix,adsmatrix}. 

If the X matrices have the block diagonal form, $X=\text{diag}(x_1,...,x_n)\,\, \text{s.t.}\,\, \sum_i^n\text{dim}(x_i)=N$ then we have a configuration of separated D0 clusters whose in-between distance is \cite{Matrix}:
\begin{equation}
    r_{ij}=|\frac{\text{Tr}(x_ i)}{\text{dim}(x_i)}-\frac{\text{Tr}(x_ j)}{\text{dim}(x_j)}|
\end{equation}

\paragraph{Spectrum}
It is known that the system without the centre of mass has bound states \cite{bound_states_1,bound_states_2,bound_states_3} in which the energy is just the centre of mass energy. These states correspond to supergraviton states. We can also describe an arbitrary number of supergravitons by considering block diagonal matrices X which correspond to block diagonal Hamiltonians of the same form with interactions between supergravitons defined by any nigh-vanishing off diagonal terms\cite{Matrix,Mxintro,adsmatrix}.

\paragraph{Supermembranes}
Aside from gravitons, we can also probe other objects of M theory such as supermembranes. It has been shown that \ref{eq:matrix_model_hamiltonian} can be obtained from the quantisation of supermembranes\cite{BFSS,Mxintro,Matrix,Taylor,supermembranes}. We would like to illustrate a heuristic hint for this. Were we to consider a single D0-brane whose coordinates are Abelian, it would only couple to a 1-form. However, when considering N, non-abelian D0-branes, the action \ref{eq:action} also includes terms of the form:
\begin{equation}
    \iota_X \iota_XA_{(1)}\wedge B
\end{equation}
Which allow the D0-branes to act like a D2-brane by effectively coupling to a 3-form. This is only possible because the coordinates X are non-abelian\footnote{This is known as the Myers effect\cite{Myers_1999}}.

There are also several interesting interpretations of the Matrix Model if one views its relation with M Theory through the AdS/CFT correspondence. For more, refer to \cite{adsmatrix}.